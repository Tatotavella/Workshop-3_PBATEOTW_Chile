\documentclass[11pt]{amsart}
\usepackage{geometry}                % See geometry.pdf to learn the layout options. There are lots.
\geometry{letterpaper}                   % ... or a4paper or a5paper or ... 
%\geometry{landscape}                % Activate for for rotated page geometry
%\usepackage[parfill]{parskip}    % Activate to begin paragraphs with an empty line rather than an indent
\usepackage{graphicx}
\usepackage{amssymb}
\usepackage{epstopdf}
\usepackage{url}
\DeclareGraphicsRule{.tif}{png}{.png}{`convert #1 `dirname #1`/`basename #1 .tif`.png}


\begin{document}
\begin{center}
  \LARGE 
  3SPN.2 LAMMPS Implementation \par \bigskip

  \normalsize
  Daniel M. Hinckley\textsuperscript{1} and Juan J.  de Pablo\textsuperscript{2}  \par \bigskip

  \textsuperscript{1}Department of Chemical and Biological Engineering, University of Wisconsin-Maidon \par
  \textsuperscript{2}Institute for Molecular Engineering, University of Chicago\par \bigskip

  \today
\end{center}

\section{Introduction}

3SPN.2 is an improvement on the previous version of the
\textbf{3}-\textbf{S}ite-\textbf{P}er-\textbf{Nucleotide} (3SPN) coarse-grained
DNA model.  Key improvements include the replacement of G\={o}-like
interactions with angle-dependent potentials, a reduction in the magnitude of
the explicit charge on the phosphate sites, and modification of the dihedral
potential to increase the flexibility of ssDNA.  These improvements remedy a
number of limitations that were identified by the members of the de Pablo group
and others.  The resulting model can capture the persistence lengths of both
ssDNA and dsDNA, proper melting temperatures for duplexes and hairpins, and has
stable major and minor grooves.  For further details please see Hinckley, D. H.
et al. J. Chem. Phys. 139,144903 (2013).

The 3SPN.2 coarse-grained model has been implemented in LAMMPS as a user
package.  This \texttt{USER-3SPN2} package contains this documentation file,
sample .in input files for serial and parallel tempering calculations, a folder
\texttt{DSIM\_ICNF} that contains a configuration generator, and the source
file to be added to the LAMMPS source.  The following sections explain the
particulars of the LAMMPS implementation, how to generate initial
configurations for B--DNA, steps for running and visualizing trajectories, and
limitations to this implementation.  Also included are instructions for
compiling the serial version of LAMMPS with 3SPN.2 from source.  If you have
any questions or problems, contact Dan Hinckley (dhinckley@wisc.edu).

\section{Implementation Details}

3SPN.2 consists of bond, angle and dihedral bonded interactions as well as
numerous nonbonded interactions. For functional forms of these potentials, we
refer the user to the model publication in JCP.  The bonded interactions are
implemented using existing LAMMPS potentials or slight modifications thereof.
Bonds and angles are modeled using the \emph{class2} and \emph{harmonic} bond
and angle styles, respectively.  The Gaussian well dihedral potential is
modeled using a new \emph{3spn2} dihedral style 
(\texttt{dihedral\_3spn2.cpp}).  The
base-stacking nonbonded interactions currently only occur between predetermined sets of sites.
As such, they are implemented as a modified angle potential in
\texttt{angle\_3spn2\_stacking.cpp}.  Standard harmonic angles, as well as these
``stacking'' angles are both applied through a hybrid stacking interaction.
The energies of both interactions are combined and output as \texttt{Eangle} in
the thermo output.

All remaining interactions are captured by the \texttt{pair\_3spn2.cpp}.  When
initialized, this pair style sets the value of equilibrium angles and
distances.  It also populates the arrays specifying the strength of
interactions and any modulating parameters.  Lastly, it also creates a base
pairing array that assigns flags used to determine whether or not
cross-stacking interactions are to be calculated.  This is because a  base that
is base pairing with a base at the end of a DNA strand often has no base with
which to cross-stack.  In order to determine whether or not a base is at the 5'
or 3' end of DNA, special types (types 7-14) are assigned to the bases at the
5' and 3' ends.  This is similar to the notation used in all-atom forcefields,
with the obvious difference that in our coarse-grained representation the bases
are topologically identical.

When the 3SPN.2 pair style is computed, first it determines whether or not the
pair of sites (i,j) are on different molecules, separated by more than 3
nucleotides for the purposes of base pairing, and if the base pairing flag is
set.  It is also determined whether or not the bases are separated by at least
5 sites.  If base pairing is to be performed, the absolute indices of the sites
i and j are used to get the indices of the neighboring sites that participate
in the angle-dependent potentials.  Then a instance of the BasePair object is
created and populated with the instantaneous angles and distances.  That done,
member functions are called to calculate the cross stacking and base pairing
interactions.  If base pairing interactions are not present, excluded volume
interactions are calculated.   Electrostatics interactions are then calculated
and the resulting force and that from excluded volume (if applicable), are then
applied to sites i and j.

The energies are saved to a vector that allows for extraction of these energies
using a compute.  The \texttt{3spn2.in} input file displays file to screen as
follows:
\begin{verbatim}
<step> <num. bp> <Ebond> <Eangle(harmonic and stacking)> <Edihedral> ...
\end{verbatim} 
\begin{verbatim}
       <Ebp> <Ecstk> <Eelectro> <T>
\end{verbatim}

\section{Coefficients for New Styles}

The new styles have the following coefficients:
\begin{itemize}
\item \texttt{dihedral\_3spn2.cpp} - \\
\indent \texttt{ dihedral\_coeff [dihedral number] 3spn2 [$K_\phi$] [$\phi_o$] [$\sigma_{\phi,o}$]}
\item \texttt{angle\_3spn2\_stacking.cpp} - \\
\indent \texttt{ angle\_coeff [angle number] stacking/3spn2 [$\epsilon$] [$r_o$] [$\theta_o$] [$\alpha$] [$K$]}
\item \texttt{pair\_3spn2.cpp} - \\
\indent \texttt{pair\_style [$T$ (Kelvin)] [$I$ (mM)] [Short Range Cutoff]} \\
\indent \texttt{pair\_coeff 1 1 3spn2 [$\epsilon$][$\sigma$]} 
\end{itemize}

See the sample input files and the \texttt{conf\_lammps.in} generated using the
configuration generator described below for additional examples.

\section{Generating an Initial Configuration}

A configuration generator for generating B-DNA is provided in
\texttt{DSIM\_ICNF/} inside \texttt{USER-3SPN2}.  Navigate into the directory
and type \texttt{make} to build \texttt{icnf.exe}, which generates all of the
needed files for simulation and visualization.      It takes the following
arguments:
\begin{verbatim}
./icnf.exe <sequence file> <complementarity flag (0-ssDNA; 1-dsDNA)> <output directory>
\end{verbatim}
The sequence file is formatted as follows:
\begin{verbatim}
<NBPS>
<sense sequence (5'->3')>
<antisense sequence (3'->5')>
\end{verbatim}
for example,
\begin{verbatim}
14
ATATATATATATAT
TATATATATATATA
\end{verbatim}
If the antisense strand is not specified, it is assumed that you want
completely complementary DNA.  Three files are generated.  The first is
\texttt{in00\_cvmd.psf}, a topology file that can be used in VMD to visualize
the initial configuration.  The second is \texttt{in00\_conf.xyz}, an .xyz file
containing the coordinates of each coarse-grained site.  The last file is
\texttt{conf\_lammps.in}, the topology file used in LAMMPS.

After running \texttt{incf.exe}, it is possible to visualize the configuration
in VMD.  This is done with the following command:
\begin{verbatim}
vmd in00_cvmd.psf in00_conf.xyz
\end{verbatim}
To capture the correct excluded volume of each site, go to
Extensions$>$TkConsole and type \texttt{source path/to/spheres.vmd}.    The
\texttt{spheres.vmd} file is also found inside \texttt{USER-3SPN2}.  Then, if
you set the visualization style to VDW, you will have the correct excluded
volume of the sites. 

\section{Input Files}

Included in USER-3SPN2 are  \texttt{3spn2.in}, \texttt{3spn2\_restart.in} and \texttt{3spn2\_temper.in},
the input files for serial, restarted, and parallel tempering simulations, respectively.
Inside each input file, the temperature and ionic strength are specified in
Kelvin and mM, respectively.   Each file also reads the topology file,
\texttt{conf\_lammps.in}.  It must be in the same working directory as the
input file.

\section{Generating a Movie}

After performing a simulation, it is easy to visualize its trajectory in VMD
with the following command:
\begin{verbatim}
vmd in00_cvmd.psf traj.xyz
\end{verbatim}
As before, source the .vmd script to use the correct excluded volume.   If you
would like to render a particular frame of the trajectory, it can be done by
going to File$>$Render.  To make to a movie, go to
Extensions$>$Visualization$>$Movie Maker.   Most of the time it is desirable to
only render a fraction of the total frames.   The other frames can be dropped
by right clicking on the trajectory in the main window.

\section{Performing Parallel Tempering Calculations}
3SPN.2 incorporates a temperature-dependent dielectric that must be varied with temperature.
To account for this, the pair style \emph{3spn2} takes a temperature fix as an optional 4th argument when performing parallel tempering calculations. 
An example is include below:
\begin{verbatim}
variable T world 300.0 315 330 345 360 375
...
pair_style      3spn2 ${T} ${salt} 18.0 tempfix
...
fix tempfix all langevin ${T} ${T} 1000 ${random}
...
temper 100000 1000 ${T} tempfix 0 ${random}
vmd in00_cvmd.psf traj.xyz
\end{verbatim}
Note that the temperature is defined using a world-type variable and that this
variable is used in the pair style and the Langevin thermostat fix.  The temper
command, which requires the REPLICA package to run, takes both the temperature
and the thermostat fix as arguments.   For a complete example, see
\texttt{3spn2\_temper.in}. To run the simulation, issue the command
\begin{verbatim}
mpirun -np 6 ../../lmp_openmpi -partition 6x1 -in 3spn2_temper.in
\end{verbatim}


\section{Limitation to LAMMPS Implementation}

There are a number of limitations to the LAMMPS implementation that each user
should be aware of.  These are enumerated below:
\begin{enumerate}
\item 3SPN.2 is performed in an implicit solvent.  Consequently, the simulation
box is mostly empty.  This makes parallelization via LAMMPS's spatial
decomposition horribly ineffective.  3SPN.2 will not scale well on multiple
processors unless parallel tempering is being performed with each box being run
by one processor.
\item The virial (fdotr) is not currently being calculated properly.
\item The Langevin integrator is specified with a default damping constant.
This damping constant has not been rigorously optimized.
\end{enumerate}

\section{Compiling the source}\label{Compile}

The following instructions should be sufficient to compile the serial version
of LAMMPS with 3SPN.2.

\begin{verbatim}
svn co svn://svn.icms.temple.edu/lammps-ro/trunk mylammps
cp -r USER-3SPN2 mylammps/src
cd mylammps/src
cd STUBS
make
cd ..
make yes-MOLECULE
make yes-CLASS2
make yes-USER-3SPN2
make serial
\end{verbatim}
\end{document}  
